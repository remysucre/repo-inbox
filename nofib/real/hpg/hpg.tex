% hpg.tex - main LaTeX file for printing HPG

% @(#)hpg.tex   1.9 dated 92/07/20 at 17:30:35

% Crown Copyright 1992

\documentstyle[a4wide]{article}

\title{The Haskell Program Generator}
\author{N D North\\
        National Physical Laboratory\\
        Teddington, TW11 0LW, UK.\\
        {\tt ndn@seg.npl.co.uk}}

% Some macros lifted from elsewhere to make this more standalone.
\makeatletter
% INLINE PROGRAM CODE
%
% \prog{foo} sets its argument in typewriter font.
\def\prog#1{\ifmmode\mbox{\tt #1}\else{\tt #1}\fi}

% NEWVERBATIM (from iso.sty)
%
% \newverbatim{foo} creates a new environment, foo, which behaves exactly
% like the verbatim environment except that it is delimited by
% \begin{foo} ... \end{foo}.
% See the VERBATIM section of latex.tex for the inspiration behind this.
%
\def\newverbatim#1{\expandafter\def\csname #1\endcsname{%
\@verbatim \frenchspacing\@vobeyspaces \csname @x#1verbatim\endcsname}
\expandafter\let\csname end#1\endcsname=\endtrivlist
\new@xverbatim{#1}}

\begingroup \catcode `|=0 \catcode `[= 1
\catcode`]=2 \catcode `\{=12 \catcode `\}=12
\catcode`\\=12
|gdef|new@xverbatim#1[
|expandafter|def|csname @x#1verbatim|endcsname##1\end{#1}[##1|end[#1]]]
|endgroup
\makeatother

\newverbatim{haskell}
\def\HPG{{\sc HPG}}

\begin{document}
\maketitle

% intro.tex - introduction to the HPG

% @(#)intro.tex 1.9 dated 92/07/20 at 17:30:36

% Crown Copyright 1991

\section*{Introduction}

This report describes the Haskell Program Generator (\HPG) and its
implementation in Haskell~\cite{hudak}.
The \HPG\ is a program for generating random programs in a functional
programming language in order to test its compiler.
It is designed to be easily converted to produce programs in any of
the functional languages in the family including Haskell, Miranda,
ML and Hope; the version documented here generates programs in, and is
written in, Haskell.

The document consists of a section giving an overview of the \HPG,
a section for each of the system's program modules, and finally some
results and conclusions.
Further description of the module sections is contained in
section~\ref{overview}.

The report is written in the so-called literal, or inverse comment, style.
This means that lines beginning ``\prog{> }'' are lines of program code,
while all other lines are comments.
This enables the same files to be typeset in the form of this report, and
to be directly compiled by a Haskell compiler into executable code.
For those who would like the original files, they are available from the
author by electronic mail at the address \prog{ndn@seg.npl.co.uk}.


\section{Overview}
\label{overview}

Random program generators, such as the \HPG, date back to the Pascal
Program Generator~\cite{wichmann} and include, more recently, the Ada
Program Generator~\cite{austin}.
They generate pseudo-random self-checking programs for testing language
compilers.
The program generation process consists of producing a complex expression
whose value is known, and then writing a program which evaluates the
expression and checks that the value obtained is correct.
The programs generated are far more convoluted than any user, or even
a human tester, would write, but they uncover errors that can occur in
quite straightforward programs.

A random program generator is a useful adjunct to a language test suite
as it can generate many more test cases than are in a suite, and their
random nature prevents implementations from being ``tuned'' to pass the
tests.

In generating a random program the \HPG\ proceeds in a number of phases:
\begin{enumerate}
\item Generate a number of random algebraic type declarations.
\item Generate a number of declarations of random values of the types from the
    previous phase.
\item Generate random expressions which evaluate to the random
    values from the previous phase.
\item Generate a program which tests the random values for
    equality with their corresponding expressions.
\item Print the type declarations, value declarations and test program.
\end{enumerate}

The user can supply values for: the random number generator seeds, the
numbers of type declarations and expressions generated, and the complexity
of those declarations and expressions.
See section~\ref{genprog} for details of the \HPG's parameters.

The \HPG\ system consists of a number of modules, each of which is described
in a section of this report.
The modules, in order, are:
\begin{description}
\item[Types] Declarations of types used throughout the \HPG.
\item[Env] The environment used by the \HPG\ to record the current state of
    the random number generator and details of the types, values etc
    that have been generated.
\item[GenType] Generates random algebraic datatypes.
\item[GenVal] Generates random values of the algebraic datatypes.
\item[GenExp] Generates an expression evaluating to each of the random
    values --- this is the major part of the \HPG.
\item[Main] Contains the top-level of the \HPG\ and orchestrates
    the other modules to create the complete system.
\item[Config] An appendix containing reconfigurable and language-dependent
    constants.
    If the \HPG\ is to be used for generating programs in another
    language, such as Miranda, then this is where most of the alterations
    should occur.
\item[Utils] An appendix of generally useful functions used throughout
    the \HPG.
\end{description}


\input{Types.lhs}

\newpage
\input{Env.lhs}

\newpage
\input{GenType.lhs}

\newpage
\input{GenVal.lhs}

\newpage
\input{GenExp.lhs}

\newpage
\input{Main.lhs}

\newpage
% outro.tex - conclusion and references for the HPG

% %W% dated %E% at %U%

% Crown Copyright 1992

\section{Results and conclusion}

\subsection{Results}
Various errors detected.

Used as test program for new Glasgow compiler.

\subsection{Extensions}
Easily done.

\subsection{Conclusions}
It is a good thing  - easy to extend.

\section*{References}

\begin{thebibliography}{North91}
\bibitem[AWW91]{austin}
    S M Austin, D R Wilkins \& B A Wichmann.
    {\it An Ada Program Test Generator.}
    TriAda Conference Proceedings, ACM, October 1991.
\bibitem[HPW*91]{hudak}
    P Hudak, S Peyton Jones, P Wadler et al.
    {\it Report on the Functional Programming Language Haskell, Version 1.1.}
    Department of Computing Science, University of Glasgow, August 1991.
\bibitem[North91]{north}
    N D North.
    {\it Mkhprog --- a Command Line Parser Generator.}
    Division of Information Technology and Computing, National Physical
    Laboratory, May 1991.
\bibitem[WD89]{wichmann}
    B A Wichmann \& M Davies.
    {\it Experience with a Compiler Testing Tool.}
    NPL Report DITC 138/89, National Physical Laboratory, March 1989.
\bibitem[WH82]{wichmann.random}
    B A Wichmann \& I D Hill.
    {\it A Pseudo-Random Number Generator.}
    NPL Report DITC 6/82 National Physical Laboratory, June 1982.
\end{thebibliography}


\appendix

\input{Config.lhs}

\newpage
\input{Utils.lhs}

\end{document}
